\documentclass[9pt,final,hyphenatedtitles]{papertex}
\usepackage{luatextra}
\usepackage{geometry}
\usepackage{polyglossia}
\usepackage{ulem}
\usepackage{framed}
\setmainlanguage{french}
\setmainfont{Latin Modern Roman}

\geometry{margin={0.4in,0.7in}}

\title{Robots: où en sommes-nous?}
\author{Pierre-Emmanuel Novac}

\begin{document}

\begin{news}{3}
	{Robots humanoïdes: que nous réserve le futur?}
	{Des robots d'un genre nouveau: bébés, soldats, clones, partenaires… Toutes les formes existeront. Polyvalents et destinés à accompagner ou à remplacer l'Homme, ils peupleront bientôt le monde entier à nos côtés. Comment percevrons-les-nous et comment seront-ils intégrés dans notre société?}
	{TECHNOLOGIES}
	{1}
	
	
	
\authorandplace{Pierre-Emmanuel Novac}{}
\vspace*{2.5mm}
 Les robots ont désormais leur place dans notre vie quotidienne. Ils existent déjà sous différentes formes et sont utilisés dans des domaines variés comme l'industrie où ils effectuent des tâches dangereuses de peinture des carrosseries ou dans la médecine pour des opérations de précision. Ils ne sont pas dotés d'intelligence artificielle et ne sont destinés qu'à une utilisation spécifique. Mais les projets de robots humanoïdes font de plus en plus parler d'eux et soulèvent de nombreuses questions, à la fois techniques et éthiques.

\columntitle{lines}{Des robots à l'apparence humaine}

La disponibilité sur le marché de véritables robots humanoïdes polyvalents n'est qu'une question de temps. Un certain nombre de robots pensés sur un modèle humain sont déjà utilisés pour accompagner les ouvriers, comme c'est le cas dans les usines \textit{Glory} où ils sont considérés comme de véritables collègues. Ces robots dénomés \textit{NEXTAGE} ont l'avantage de pouvoir être utilisés en permanence sans se fatiguer ni se plaindre, mais ils ne peuvent pas effectuer de tâches délicates sollicitant les cinq sens et leur productivité ne dépasse pas 80\% de celle d'un humain.




\textit{Kawada}, fabriquant de ce robot, en produit un autre: \textit{HRP2}. Déjà utilisé dans de nombreux laboratoires de recherches, il peut se déplacer grâce à deux jambes motorisées. Le \textit{Joint Robotics Laboratory}, collaboration entre chercheurs français et japonais, a pour projet de piloter ce robot rien que par la pensée. Un doute s'installe quant à l'utilisation de ces outils qui pourrait devenir malveillante… Mais cela peut permettre à un tétraplégique de gagner en autonomie. Objectif du Pr Kheddar, chef du projet: dissoudre la distance entre humains et robots.

\columntitle{lines}{Donner la vie à un robot}

Dans les laboratoires Asada, de nombreux projets sont à l'étude. Ils sont conçus pour mieux comprendre le développement humain. Pour cela, le Pr. Asada fabrique des robots sur le modèle humain. \textit{Affetto} en est un: il ressemble à un bébé. Son but: étudier le développement du langage, comprendre l'évolution des relations entre la mère et la machine. Un autre, \textit{CB2}, a servi à analyser la manière dont un bébé apprend à ramper. Il est inspiré d'un bébé de seulement un an, mais c'est insuffisant pour le Pr. Asada. Il veut remonter plus loin, aux origines de la vie. Le résultat est \textit{Fetal Robot}, dont le but est de permettre de comprendre comment un foetus de 35 semaines perçoit le monde.

\vspace*{2.5mm}
\fbox{\includegraphics[scale=0.12]{Atlas_frontview_2013}}\vspace*{1mm}\\
\textit{\hspace*{3.5mm}Atlas, robot humanoïde de sauvetage \\\hspace*{1.5cm}de Boston Dynamics}
\columntitle{lines}{Sauver des vie ou donner la mort?}

Le département de la Défense des États-Unis a lancé un concours pour créer un robot capable d'intervenir en cas de catastrophe, comme celle de Fukushima. Il doit pouvoir marcher au milieu des décombres, conduire un véhicule, utiliser un outil, ouvrir une porte, fermer une vanne. L'équipe de Matt Dedonato développe Atlas, un robot dont le comportement devra se rapprcher le plus possible de celui de l'humain afin d'évoluer de manière autonome dans notre monde. Mais, derrière la volonté de créer un robot si polyvalent pour sauver des vies, l'intention ne serait-elle pas de l'utiliser directement sur le champ de bataille? Dès lors, une importante question éthique se pose: aura-t-il le droit de tuer? Pour Ronald C. Arkin, auteur du livre \textit{Le droit de tuer chez les robots autonomes}, il faut plutôt penser à une diminution du nombre d'erreurs impliquant des civils. Tout le monde n'est pas du même avis et le débat et très vif, surtout après le rachat de Boston Dynamics, principal constructeur de ce type de robots, par Google.




\columntitle{lines}{Intégration en société}

Au sein de l'institut de recherche en télécommunication avancée au Japon est developpé Telenoid, un robot piloté à distance par l'Internet. Le but est de recréer la sensation de présence, la notion de \textit{sonzaikan} en japonais, au travers d'un robot. Son créateur, Hiroshi Ishiguro, a déjà réalisé un clone de lui-même et cherche désormais à garder le plus possible la même apparence que son robot. En viendrons-nous à idolâtrer aussi les robots? Un des aspects du shintoïsme, principale religion des Japonais, est de considérer les objets, les êtres, les notions plus abstraites comme des divinités. Pourquoi pas les robots?

\vspace*{2mm}

\setlength{\fboxsep}{0.3mm}%
\FrameSep0pt
\begin{framed}
\hspace*{-5mm}\includegraphics[scale=0.40]{2414557-robots-industriels-le-retard-francais_c}\\
\hspace*{9mm}\textit{Nombre de robots pour 10 000\\
\hspace*{8mm} salariés de l'industrie en 2014}\\


\hspace*{-2mm}\parbox{5.4cm}{La Corée du Sud et le Japon en tête avec respectivement 440 et 325 robots pour 10 000 salariés, ces pays sont parmis les précurseurs des évolutions en robotique. Leurs industries sont les plus robotisées. Parmis ces robots figurent désormais des humanoïdes comme le \textit{NEXTAGE}, encore introuvables en France.}\\
\vspace*{2mm}
\end{framed}

\columntitle{lines}{Relations avec l'Homme}

Le Pr. Gross du laboratoire de robotique de l'université d'Ilmenau en Allemagne a lancé un programme de recherche visant à développer \textit{Max}. Robot d'assistance aux personnes agées, il accompagnerait celles-ci pour lutter contre la solitude et fournir une assistance personnelle. Ce robot présent en permanence aux côtés des personnes agées, pouvant filmer et être contrôlé à distance, porte-t-il atteinte à la vie privée? Cela ne pose pas de problème à Horst Puschke, retraité expérimentant ce nouveau concept, tant qu'il peut décider de ce que peut faire le robot.

D'autre part, ils n'existent pour l'instant que comme arnaques mais rien n'empèche à une entreprise d'en créer, moyennant un certain investissement. Les robots sexuels ne sont plus si inimaginables et lorsqu'ils existeront, les relations entre humains et robots ne se distingueront plus des relations entre humains. Est-ce que ce serait aller trop loin? Difficile de répondre à cette question sans l'avoir vécu.\end{news}

\end{document}
