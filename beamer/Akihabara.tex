\documentclass{beamer}
\usepackage{luatextra}
\usepackage{polyglossia}
\usepackage{ulem}
\usepackage{framed}
\setmainlanguage{french}
\setmainfont{Latin Modern Roman}


\newcommand\image[2]{
\directlua{
local image = img.scan({filename = "#1"})

image.height = image.height * #2
image.width  = image.width  * #2

node.write(img.node(image))
}
}


\title[Akihabara]{Akihabara}
\author[]{Pierre-Emmanuel Novac}
\institute{\image{polytechnice.png}{0.05}}
\date{\today}


\usetheme{Berkeley}
\usecolortheme{seahorse}
\makeatletter
\beamer@headheight=1.5\baselineskip

\patchcmd{\insertverticalnavigation}%
{\ifx\beamer@nav@css\beamer@hidetext{\usebeamertemplate{section in sidebar}}\else{\usebeamertemplate{section in sidebar shaded}}\fi}%
{{\usebeamertemplate{section in sidebar}}}{}{}
\makeatother

\begin{document}

\section{Introduction}

\makeatletter
\begin{frame}
\frametitle{Akihabara \image{imagest.png}{0.3}}
\Large En quoi le quartier d'Akihabara est-il consacré à une population très stéréotypée?
\image{Intro.jpg}{0.17}
\tiny\textit{Akihabara}\\
Pierre-Emmanuel Novac --- \@date \hspace*{2cm}\image{polytechnice.png}{0.05}
\end{frame}
\makeatother


\begin{frame}
\frametitle{Pourquoi ce choix}
\begin{itemize}
  \item Quartier marquant et dépaysant
  \item Phénomène inconnu en France
\end{itemize}
\end{frame}

\begin{frame}
\frametitle{Sommaire}
\tableofcontents
\end{frame}


\section{Présentation du lieu}
\subsection{Histoire}

\begin{frame}
\begin{itemize}
	\item 1950 : postes radio, téléviseur et composants électroniques
	\item 1960 : électroménager
	\item 1970 : ordinateurs
	\item 1980 : jeux vidéos
	\item 1990 : informatique grand public
	\item 2000 : animation japonaise
\end{itemize}
\end{frame}

\subsection{Géographie}

\begin{frame}
\frametitle{Localisation}
\image{Tokyo_Arrondissements.png}{0.34}
\tiny\textit{Carte de Tokyo}
\end{frame}

\begin{frame}
\frametitle{Plan du quartier}
\image{Akihabara_quartier_cropped.png}{0.32}
\end{frame}

\begin{frame}
\frametitle{Accès}
\image{plan-metro-tokyo.png}{0.17}
\end{frame}

\section{Ce que j'y ai fait}

\subsection{Électronique}

\begin{frame}
\frametitle{Câbles}
\image{Cables.JPG}{0.17}
\tiny\textit{Des centaines de mètres de câbles}
\end{frame}

\begin{frame}
\frametitle{Composants électroniques}
\image{Petits condensateurs.JPG}{0.17}
\tiny\textit{De nombreux condensateurs}
\end{frame}

\begin{frame}
\frametitle{Composants électroniques}
\image{Gros condensateurs.JPG}{0.17}
\tiny\textit{Des plus gros}
\end{frame}

\subsection{Jeux}

\begin{frame}
\frametitle{Jeux d'occasion}
\image{Jeux PS2.JPG}{0.17}
\tiny\textit{Jeux vidéo d'occasion pour PlayStation 2}
\end{frame}

\begin{frame}
\frametitle{Salles d'arcade}
\image{GameCenter1.JPG}{0.07}
\tiny\textit{Un jeu de palets}
\end{frame}

\begin{frame}
\frametitle{Salles d'arcade}
\image{GameCenter2.JPG}{0.07}
\textit{Un jeu musical}
\end{frame}

\subsection{Culture manga}

\begin{frame}
\frametitle{Mangas}
\image{Mangas.JPG}{0.17}
\tiny\textit{Étalage de mangas}
\end{frame}

\begin{frame}
\frametitle{Cosplays}
\image{Cosplay.JPG}{0.17}
\tiny\textit{Cosplay issus d'un jeu vidéo}
\end{frame}

\subsection{Cafés}

\begin{frame}
\frametitle{Maid Cafés}
\image{Maid Café.JPG}{0.17}
\tiny\textit{Affiche de publicité pour un Maid Café}
\end{frame}

\begin{frame}
\frametitle{Maid Cafés}
\image{Moi_Maids.JPG}{0.17}
\tiny\textit{Maids prospectant auprès des passants}
\end{frame}

\begin{frame}
\frametitle{Neko Cafés}
\image{Neko Cafe.JPG}{0.07}
\tiny\textit{Intérieur d'un Neko Café}
\end{frame}


\section{Mon analyse}
\subsection{Points négatifs}
\begin{frame}
  \begin{itemize}
    \item Les prix
    \item L'agitation
    \item Le sexisme
  \end{itemize}
\end{frame}
\subsection{Points positifs}
\begin{frame}
  \begin{itemize}
    \item Diversité des produits
    \item Nombreux divertissements
  \end{itemize}
\end{frame}


\section{Conclusion}

\begin{frame}
  \frametitle{Conclusion}
  \begin{itemize}
    \item Quartier orienté électronique, jeux vidéo, informatique, manga
    \item Et à l'avenir?
  \end{itemize}
\end{frame}

\begin{frame}
  \frametitle{Questions}
  \Large Des questions?
\end{frame}

\end{document}
